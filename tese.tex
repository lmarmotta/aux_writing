\documentclass[12pt,twoside,a4paper]{article}

% -----------------------------------------------------------------------------
%                           PACAKGE SELECTION
% -----------------------------------------------------------------------------

% Default packages for latex to work in portuguese.
\usepackage[utf8]{inputenc}      % Dont know.
\usepackage{indentfirst}         % Paragraphs.
\usepackage[brazilian]{babel}    % Portuguese language only.
\usepackage{lingmacros}          % Other languages.

% Graphics packages.
\usepackage{float}               % Pictures, obey me !

% Bibliography.
\usepackage{hyperref}            % Link references.

% Margins and geometry.
\usepackage{geometry}
\geometry{a4paper,left=30mm,right=30mm,bottom=30mm,top=30mm}

% Math and code writing stuff.
\usepackage{amsmath}
\usepackage{listings}

% Images.
\usepackage{graphicx}
\usepackage{subcaption}


% -----------------------------------------------------------------------------
%                           HERE STARTS THE PARTY
% -----------------------------------------------------------------------------

\numberwithin{equation}{section}

\title{
    {\vspace{0mm} Minha universidade}\\
    {\vspace{70mm} \textbf{O título do trabalho vai aquí}} \\
}
\author{Leonardo Motta Maia de Oliveira Carvalho}

\date{\vfill São José dos Campos, 1 de Janeiro de 2017}


\begin{document}
\maketitle
\newpage
\tableofcontents
\bibliographystyle{plain}

\newpage
\begin{abstract}
    Aqui vai o abstract do trabalho.
\end{abstract}

\newpage

\section{Uma seção só com coisas escritas}
Aqui vai um paragrafo da seção Aqui vai um paragrafo da seçã Aqui vai um paragrafo da seçã Aqui vai um paragrafo da seçã Aqui vai um paragrafo da seçã

Aqui vai um paragrafo da seção Aqui vai um paragrafo da seçã Aqui vai um paragrafo da seçã Aqui vai um paragrafo da seçã Aqui vai um paragrafo da seçã

\section{Uma seção com coisas escritas e equações}
Aqui vai um paragrafo da seção Aqui vai um paragrafo da seçã Aqui vai um paragrafo da seçã Aqui vai um paragrafo da seçã Aqui vai um paragrafo da seçã

Aqui vai um paragrafo da seção Aqui vai um paragrafo da seçã Aqui vai um paragrafo da seçã Aqui vai um paragrafo da seçã Aqui vai um paragrafo da seçã

\begin{equation}
    A= B+C \frac{A}{B}
\end{equation}


\section{Uma seção com coisas escritas e imagens}
Aqui vai um paragrafo da seção Aqui vai um paragrafo da seçã Aqui vai um paragrafo da seçã Aqui vai um paragrafo da seçã Aqui vai um paragrafo da seçã

Aqui vai um paragrafo da seção Aqui vai um paragrafo da seçã Aqui vai um paragrafo da seçã Aqui vai um paragrafo da seçã Aqui vai um paragrafo da seçã


    % A single image.
    \begin{figure}[htb]
        \centering
        \includegraphics[width=.7\linewidth]{image1.pdf}
        \caption{The caption}
        \label{Image}
    \end{figure}

Agora eu vou citar a figura \ref{Image}.

    % Images side by side
    \begin{figure}[H]

    \begin{subfigure}{.5\textwidth}
    \centering
    \includegraphics[width=.9\linewidth]{image1.pdf}
    \caption{a}
    \label{fig:sfig1}
    \end{subfigure}%
    \begin{subfigure}{.5\textwidth}
    \centering
    \includegraphics[width=.9\linewidth]{image1.pdf}
    \caption{b}
    \label{fig:sfig2}
    \end{subfigure}

    \caption{Legenda para o conjunto de imagens}
    \label{aloha}
    \end{figure}

    Aqui foi um exeplo de multiplas figuras. \ref{aloha}

    \begin{figure}[H]

        \begin{subfigure}{.5\textwidth}
        \centering
        \includegraphics[width=.9\linewidth]{image1.pdf}
        \caption{a}
        \label{fig:sfig1}
        \end{subfigure}%
        \begin{subfigure}{.5\textwidth}
        \centering
        \includegraphics[width=.9\linewidth]{image1.pdf}
        \caption{b}
        \label{fig:sfig2}
        \end{subfigure}
        \\
        \begin{subfigure}{.5\textwidth}
        \centering
        \includegraphics[width=.9\linewidth]{image1.pdf}
        \caption{c}
        \label{fig:sfig1}
        \end{subfigure}%
        \begin{subfigure}{.5\textwidth}
        \centering
        \includegraphics[width=.9\linewidth]{image1.pdf}
        \caption{d}
        \label{fig:sfig2}
        \end{subfigure}

    \caption{Exemplo de quatro imagens juntas}
    \label{quatro}
    \end{figure}

\section{Exemplo de seção com mais de duas subseções}
Aalgo escrito aqui
\subsection{Sub}
Algo escrito aqui tbm.
\subsubsection{subsub}
Algo escrito aqui.

\section{Uma seção com códigos}

Códigos.

\lstinputlisting[language=Gnuplot,frame=single,caption={Gnuplot code},breaklines=true,numbers=left,tabsize=4,basicstyle=\small,framexleftmargin=20pt]{plot.gnu}


\section{Citando bibliografias}
Vamos aqui citar o \cite{HIRSCH_VOL2}


% This is the bib file call. Always in the end of the file please.
\bibliography{refs.bib}

\end{document}
